\documentclass[12pt]{article}
\usepackage[margin=2cm]{geometry}
\usepackage{graphicx}
\usepackage{parskip}
\usepackage{datatool}
\usepackage{cleveref}
\usepackage{amsmath}
\usepackage{amssymb}

\begin{document}
\pagenumbering{gobble}
% Usage: \dat{variable-name}
%   This prints the value of the variable to the document
\DTLloaddb[noheader, keys={thekey,thevalue}]{dat}{../../data/derived/report-summary-data.csv}
\newcommand{\dat}[1]{\DTLfetch{dat}{thekey}{#1}{thevalue}}

\section*{Markov chain Monte Carlo storage guide}%

When using Markov chain Monte Carlo (MCMC) methods to sample $\theta^{(1)},\dots, \theta^{(m)} \in \mathbb{R}^p$, sequentially, from the joint posterior
$\pi(\theta)$, the samples should be stores in pre-allocated storage.
Relevant options are using either R data frames or R matrices, and either storing the samples row-wise or column-wise.
If storing the samples column-wise, the storage should be transposed after being filled, and if using matrices, they should be
converted back to data frames at the end.
The use of R data frames for this task is heavily discouraged, as the execution time scales poorly with the number of samples, $m$,
as evident from \cref{fig:time}.

\begin{figure}[htp]
\begin{center}
    \includegraphics[scale=1.0]{../../outputs/figures/mcmc-execution-times.pdf}
\end{center}
\caption{%
    For each of the four storage allocation strategies described above, the matrix or dataframe was filled sequentially using a
    dummy MCMC sampler. Both the storage initialisation, the sampling and the transformation and conversion at the end, if applicable,
    was timed. This was done for $m$-values logarithmically spaced between 1 and $m_{max}=\protect\dat{max_m}$. For all the simulations,
    the dimensions of the posterior distribution was set to
    $p=\protect\dat{num_model_parameters}$.
    The confidence intervals are asymptotic normal 95\% intervals, based on repeating the simulation
    \protect\dat{num_simulation_repeats} times.
}
\label{fig:time}
\end{figure}

Instead, the storage should be pre-allocated using either a $p \times m$ or a $m \times p$ R matrix, and should then be filled
column-wise or row-wise, respectively. The matrices should then be converted to a data frame after being filled, and if filling column-wise, the matrix should be transposed before converting to a data frame.
There is no statistically significant difference between the two matrix approaches as their execution times for $m=\dat{max_m}$ samples
is very similar:
\begin{center}
    \begin{tabular}{c|c}
        \textbf{Storage strategy} & \textbf{95\% asymptotic normal confidence interval for execution time} \\
        \hline
        Matrix row-wise & $\dat{mat_row_avg}\pm\dat{mat_row_95}$ seconds \\
        Matrix column-wise & $\dat{mat_col_avg}\pm\dat{mat_col_95}$ seconds
    \end{tabular}
\end{center}

Therefore, either of the two matrix option are acceptable.
\end{document}
